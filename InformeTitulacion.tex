\documentclass[a4paper,12pt]{article}
\usepackage{apacite}
\usepackage{graphicx}


\title{\includegraphics[scale=2]{logo.png} \\Informe del dictado de la cátedra integradora \\ Metodología de la Investigación}
\begin{document}
\author{Ing. Stalin Francis Q. M.Sc.\\ Docente de la cátedra}
\maketitle
\section{Antecedente}
\label{sec:antecedente}


El 3 de marzo del 2022 la dirección de la carrera en Tecnología de la Información de la Facultad de Ingeniería me notifica para colaborar con el dictado de la cátedra integradora llamada Metodología de la Información dirigida a los estudiantes que se inscribieron en el paralelo B para el proceso de titulación; esta asignatura se dictó en las fechas comprendidas entre el 10 de marzo al 8 de abril del 2022, con una duración de 1 mes, en horarios de jueves y viernes de 16h:00 a 18h:00.\\

La dirección de carrera mediante oficio, me hizo la entrega de un listado con 52 estudiantes que se habían inscrito en el paralelo antes indicado; después de los cual procedí a conseguir y elaborar el material didáctico con el cual dictaría la asignatura; este material consitió en la guía de aprendizaje y las presentaciones respectivas.\\

En vista de que aun la universidad se encontraba en una modalidad de trabajo especial como es la enseñanza en linea, y su plataforma de trabajo oficial era la llamada ClassRoom proporcionada por la empresa Google, se procedió a crear el respectivo curso virtual y hacer la respectiva configuración del mismo que consistió en llenar el microcurrículo, colgar los recursos o materiales didácticos, programar las actividades de aprendizaje y hacer la invitación a los estudiantes para que se unan al curso.\\

En el resto de este documento se detalla todos los pormenores del desarrollo de la cátedra.


\section{Metodología}
\label{sec:metodologia}

Durante el desarrollo del curso se utilizó la metodología de aprendizaje llamada Aula invertida, donde se le asignó a los estudiantes textos, videos y otros contenidos para revisar fuera de clase. Se utilizó la plataforma Google ClassRoom para asignar actividades de aprendizaje y se utilizó  el software de videoconferencias Google meet para la clases sincrónicas.


\section{Microcurrículo}
\label{sec:microcurriculo}

Los temas del microcurrículo son los siguiente:
\begin{enumerate}
\item SEMANA-01: Planteamiento del problema.
\item SEMANA-02: Marco teórico.
\item SEMANA-03: Diseño metodológico.
\item SEMANA-04: Propuesta.
\item SEMANA-05: Conclusiones y recomendaciones.
\end{enumerate}

\section{Actividades de aprendizaje}
\label{sec:activ-de-aprend}

Se llevaron a efecto 3 actividades de aprendizaje las cuales fueron calificados con valores que van entre 0 y 100\%.

\subsection{Actividad A1}
\label{sec:actividad-a1}

\paragraph{Nombre de la actividad:}

Elaboración del planteamiento del problema, preguntas de investigación, Objetivos generales, específicos y Justificación.

Consistió en elaborar la parte inicial del proyecto de investigación, esto comprendía el planteamiento del problema, preguntas de investigación, objetivos generales y específicos ademas de la justificación.

\subsection{Actividad B1}
\label{sec:actividad-b1}

\paragraph{Nombre de la actividad:}

Análisis del problema, mediante el árbol de problema.\\

Esta actividad consistió en dos subactividades.\\

\begin{itemize}
\item Un video donde el estudiante despúes de una profunda lectura crítica de la unidad 2 del material proporcinado, explique con sus propias palabras los comprendido y lo relacione con el trabajo realizado en la actividad anterior.
\item Elaborar un árbol de problema, del problema propuesta en la actividad anterior a esta.
\end{itemize}


\subsection{Actividad C1}
\label{sec:actividad-c1}

\paragraph{Nombre de la actividad:}

Redacción del marco teóricos del problema planteado.\\

En esta última actividad el estudiante elabora el marco teórico del planteado en la actividad A1 y analizado con el árbol de problema en la actividad B1; se le pidió que en la redacción utilice fuentes confiables (libros y artículos científicos) y sea exigente al aplicar las normas APA.

\section{Conclusiones}
\label{sec:conclusiones}

\begin{enumerate}
\item La cátedra se dictó en modalidad online, gracias a la tecnología de la información y comunicación la cual funciono sin ningún contratiempo ayudando a que el proceso de enseñanza aprendizaje fluya con total normalidad.
\item A los estudiantes se les pudo detectar fuertes vacíos en los conocimientos y habilidades para plantear un problema, plantear objetivos tanto generales como específicos y aplicar normas APA para la redacción del marco teórico.
\item Existe un marcado habito de entregar trabajos deficientes esperando que el docente haga correcciones a errores que los estudiantes saben de antemano que existen.
\end{enumerate}


\section{Recomendaciones}
\label{sec:recomendaciones}

\begin{enumerate}
\item Las asignaturas de la carrera deben reforzar en todos sus niveles hábitos de lectura y de investigación de fuentes confiables como libros y artículos.
\item Debe implementarse algún programa de capacitación continua en metodología de la investigación que exija al estudiantes refrescar conocimientos al inicio de la carrera, a la mitad de la carrera y al final de la carrera.
\item Debe implementarse un programa de escritura de artículo científicos durante la carrera. 
\end{enumerate}



\bibliographystyle{apacite}
\bibliography{referencia}

\end{document}


%%% Local Variables:
%%% mode: latex
%%% TeX-master: t
%%% End:
